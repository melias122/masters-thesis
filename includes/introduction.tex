
Výber diplomovej práce \textit{Lúštenie historických šifier na GRIDe} bol podmienený osobným zájmom o túto problematiku.
Hlavným cieľom tejto práce je vytvorenie nástrojov na kryptoanalýzu klasických šifier v gridovom prostredí SIVVP.
V tejto práci implementujeme útok pomocou paralelných genetických algoritmov.
Tematikou paralelných výpočtov sa zaoberala aj práca \cite{pea} a čiastočne aj práca \cite{ev}, z ktorých sme čerpali niekoľko základných poznatkov o paralelných genetických algoritmoch.
Ďalším cieľom tejto práce je vyvinúť vývojové prostredie slúžiace pre ďalších potenciálnych lúštiteľov, ktorí by mohli potrebovať ku svojej práci gridové prostredie.

Štruktúra diplomovej práce sa skladá zo šiestich kapitol.
Prvá kapitola slúži ako teoretický úvod do problematiky klasických šifier vychádzajúc zo skrípt \cite{ks}, počnúc históriou, charakteristikou a počítačovým lúštením klasických šifier. V tejto kapitole predstavíme aj niektoré základné útoky.

Druhá kapitola práce sa zameriava na predstavenie gridu \textit{hpc.stuba.sk}, pričom naším cieľom bude predstaviť základy práce so spomínaným prostredím.
Okrem toho sa budeme snažiť poukázať na poskytovanú funkcionalitu tohto prostredia a predstavíme niekoľko základných typov úloh.

V tretej kapitole si uvedieme základné princípy MPI komunikácie, ktoré budeme využívať pri medziprocesorovej komunikácii na gride.
Taktiež poskytneme aj niekoľko principiálnych ukážok komunikácie a spomenie si dva základné problémy súvisiace s výmenou dát pomocou MPI.

Nasledujúca kapitola pozostáva zo stručného návrhu a požiadaviek kladených na vývojové prostredie.
Táto kapitola má praktické zameranie a predstavíme si v nej použité technológie, základ vývojového prostredia ktorým je skript build.sh.
V nasledujúcom kroku predstavíme vývoj a implementáciu paralelných genetických algoritmov. Táto kapitola má slúžiť aj ako manuál
pre ďalších používateľov tohto prostredia.

V piatej kapitole vykonáme experiment s genetickými algoritmami a ukážeme prístup zvolený pri paralelizácii programu.

Záverečná kapitola pozostáva z ďalšieho experimentu súvisiaceho s paralelnými genetickými algoritmami, pri ktorom vychádzame z kapitoly päť.

