%% Výber diplomovej práce Lúštenie historických šifier na GRIDe bol podmienený osob-
%% ným zájmom o túto problematiku. Hlavným cieľom tejto práce je zaoberanie sa lúšte-
%% ním monoalfabetickej substitúcie pomocou paralelných genetických algoritmov v grido-
%% vom prostredí. Tematikou paralelných výpočtov sa zaoberala aj práca [1], z ktorej sme
%% čerpali niekoľko základných poznatkov o paralelných genetických algoritmoch. Ďalším cie-

Ambíciou diplomovej práce bolo vytvorenie nástrojov na kryptoanalýzu klasických šifier v gridovom prostredí SIVVP.
Tento krok bol úspešne zrealizovaný. Vytvorili sme vývojové prostredie, ktoré zjednodušuje prácu a vývoj v lokálnom prostredí
a prostredníctvom ktorého je možné: vytvárať projekty,  synchronizovať ich s gridom a spúšťať lokálne a tiež na gride.
V nami vyvinutom vývojovom prostredí sme vytvorili modul pre paralelné genetické algoritmy,
ktorý si ďalší lúštitelia môžu prispôsobiť podľa vlastných potrieb.

Tento modul sme si aj my upravili podľa vlastných potrieb a implementovali sme útok na monoalfabetickú substitúciu pomocou PGA.
V prvom experimente sme hľadali vhodné schémy a nastavenia GA, pričom sme využili distribúciu parametrov na 96 uzloch gridu.
Tento experiment trval 34 hodín. Výsledkom experimentu bolo, že sedem z desiatich schém dosahovalo 80\% úspešnosť pri textoch
dĺžky 500 znakov a viac ako 90\% úspešnosť pri textoch dlhších ako 1000 znakov.

Schémy z prvého experimentu sme použili v nasledujúcom experimente s PGA.
Na týchto schémach sme skúmali vplyv štyroch rôznych topológií a troch migračných časov, a ich úspešnosť pri lúštení
monoalfabetickej substitúcie.
TODO: vysledky


%% Zaver
%%
%% Spolupráca s Petrom Javorkom
%% Spotrebovaný čas/peniaze
