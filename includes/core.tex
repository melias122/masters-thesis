\section{Klasické šifry}
V tejto kapitole sa budeme zaoberať históriou a stručným prehľadom klasických šifier.
Spomenieme si aj niektoré základné útoky na klasické šifry. 

\subsection{História}
História klasických šifier a utajovania písomného textu je pravdepodobne tak stará ako samotné písmo.
Písmo, v podobe akej ho poznáme a používame dnes, pravdepodobne pochádza asi spred 3000 rokov pred Kristom a za jeho objaviteľov sa považujú
Feničania.
V niektorých prípadoch predstavovalo už použitie písma utajenie samotného textu.
Príkladom môžu byť Egyptské hieroglyfy alebo klinové písmo používané v Mezopotámii.
Iným príkladom môžu byť semitské jazyky, ktoré sú charakteristické používaním iba spoluhlások bez použitia samohlások,
pretože tie zaviedli až Aremejci a po nich následné Gréci aby pomocou nich boli schopný rozlíšiť jazyky \cite{ks}.
Aj diakritika ako taká má schopnosť rozlišovať významy slov, čo si ale až do 15.storočia nikto nevšímal,
až pokiaľ ju Arabi nezačali používať pri kryptoanalýze rôznych šifier.

Z historického hľadiska nie je možné presne zoradiť ako jednotlivé šifry vznikali, pretože súčasne vznikali na viacerých miestach sveta.
Komunikácia a s ňou spojené sírenie informácii nebolo také rýchle ako dnes, až do roku 1440 keď Johan Guttenberg vynašiel kníhtlač,
čo zjednodušilo výmenu a uchovávanie informácii.
% (TODO: pridať utajovanie informácie)

Ku kryptografii ako aj k rôznym iným vedným disciplínam prispelo v minulosti staré Grécko.
Jedným z najvýznamnejších príspevkov starých Grékov bolo široké rozšírenie abecedy a písomného prejavu.
Gréci písmo prebrali od Feničanov, ktorí na rozdiel od Egypťanov používali jednoduchšie písmo.

V Európe vďaka rozšíreniu abecedy začali vznikať aj prvé šifry, medzi ktoré patrí napríklad Cézarova šifra, ktorá vznikla v Rímskej ríši.
Iným príkladom môže byť transpozičná šifra skytalé, ktorá bola používaná v Sparte.

Pád Rímskej ríše spôsobil úpadok kryptografie, ktorý trval až do obdobia stredoveku. Typickým znakom kryptografie v tomto období bolo
napríklad písanie odzadu, alebo vertikálne, používanie cudzích jazykov, alebo vynechávanie samohlások \cite{ks}.

V stredoveku, kvôli bojom medzi pápežmi Ríma a Avignonu, bola kryptografia zdokonalená a začali sa používať rôzne kódy a nomenklátory.
Ich charakteristickým znakom bolo zamieňanie písmen alebo nahradzovanie mien a titulov osôb v správach.
V tomto období zabezpečovanie utajenia správ pokročilo až na takú úroveň, že na doručovanie správ boli použitý špeciálne vycvičení kuriéri.

V prvej polovici 20. storočia ľudia, ktorí pracovali v oblasti utajovanej komunikácie verili, že na to aby bola zabezpečená utajovaná komunikácia musí byť utajený kľúč a okrem neho aj šifrovací algoritmus. Toto ale odporovalo Kerckhoffovmu princípu, ktorý hovorí že: \textquote{Bezpečnosť šifrovacieho algoritmu musí závisieť výlučne na utajení kľúča a nie algoritmu}. Okrem toho sformuloval aj niekoľko požiadaviek na kryptografický systém, medzi ktoré patria:
\begin{enumerate}
\item systém musí byť teoreticky, alebo aspoň prakticky bezpečný
\item narušenie systému nesmie priniesť ťažkosti odosielateľovi a adresátovi
\item kľúč musí byť ľahko zapamätateľný a ľahko vymeniteľný
\item zašifrovaná správa musí byť prenášateľná telegrafom
\item šifrovacia pomôcka musí byť ľahko prenosná a ovládateľná jedinou osobou
\item systém musí byť jednoduchý, bez dlhého zoznamu pravidiel, nevyžadujúci nadmerné sústredenie
\end{enumerate}
Tieto princípy sú popísané v pôvodnej publikácii od Kerckhoffa \cite{kerckhoff}.

Existovala ale aj iná skupina vedcov, medzi ktorých patril aj Lester S. Hill, ktorý si uvedomoval že kryptológia je úzko spätá z matematikou.
V roku 1941 si na Hillových prácach zakladal A. Adrian Albert, ktorý pochopil, že v šifrovaní je možné použiť viacero algebraických štruktúr.
Neskôr toto všetko usporiadal a zdokonalil Claude E. Shannon, čo možno považovať za ukončenie éry klasických šifier \cite{ks}.

% \todo{Možno pridať/spomenúť steganografiu.}

\subsection{Charakteristika}
Na rozdiel od moderných šifier, ktoré sa používajú dnes, sú tie klasické rozdielne v niektorých hlavných črtách.
Môžeme spomenúť niekoľko:
\begin{itemize}
\item Šifrovanie a dešifrovanie klasickej šifry možno realizovať zväčša pomocou papiera a ceruzky alebo nejakej mechanickej pomôcky.
\item V dnešnej dobe aj vďaka rozšírenému použitiu počítačov stratila väčšina týchto algoritmov svoj význam.
\item Utajuje sa algoritmus a aj kľúč a neuplatňuje sa Kerckhoffov princíp.
\item Na rozdiel od moderných šifier sa používajú malé abecedy.
\item V klasických šifrách je otvorený text, zašifrovaný text a kľúč v abecede reálneho jazyka, pričom v moderných šifrách sa používa binárne kódovanie.
\item Na klasické šifry sa zväčša dá použiť štatistická analýza. 
\end{itemize}
Z spomenutých charakteristík existujú aj výnimky. Napríklad pri Vigenerovej šifre sa algoritmus neutajoval. To platí aj pre Vernamovu šifru, ktorá okrem toho používa navyše binárne znaky. Vernamova šifra je perfektne bezpečná v podľa Shannonovej teórie \cite{ks}.

Klasické šifry môžeme rozdeliť do niekoľkých základných kategórii:
\begin{itemize}

\item \textbf{Substitučné šifry.}
  V prípade že šifra permutuje znaky zdrojovej abecedy, hovoríme o monoalfabetickej šifre.
  Ako príklad možeme uviesť šifru Atbaš prípadne Cézarovu šifru, alebo iné.
  V inom prípade ak sa aplikuje viacero permutácii podľa polohy znaku v otvorenom texte, tak hovoríme o polyalfabetickej šifre.
  Príkladom je Vigenerova šifra. Daľsím prípadom je polygramová šifra, kde sa z otvoreného textu najprv vytvoria bloky,
  na ktoré sa potom aplikuje nejaká permutácia.

\item \textbf{Transpozičné šifry.}
  Transpozičné šifry sú vlastne blokové šifry, ktoré pri šifrovaní a dešifrovaní aplikujú pevne zvolenú permutáciu na každý blok
  otvoreného/zašifrovaného textu. Od polyalfabetickej šifry sa líši v poradí vykonávania operácii.
  
\item \textbf{Homofónne šifry.}
  Homofónne šifry sú šifry, ktoré majú znáhodnený zašifrovaný text. Tieto šifry sa snažia zabrániť frekvenčnej analýze textu. 
  
\item \textbf{Substitučno-permutačné šifry.}
  Ak aplikujeme viacero substitučný a permutačných šifier na otvorený text tak hovoríme o substitučno-permutačných šífrách.
  Šifrovanie prebieha tak, že blok otvoreného textu sa rozdelí na menšie bloky, na ktoré je potom aplikovaná substitúcia, a permutácia,
  ktorá sa aplikuje na celý blok. Substitúcia zabezpečuje konfúziu a permutácia difúziu.
  
\end{itemize}

\subsection{Útoky}
% \todo{ks 2.3}
\subsubsection{Hrubou silou}
Útok hrubou silou (bruteforce) je typ útoku, ktorý sa snaží zlomiť kľúč tak, že sa prehľadáva celý priestor kľúčov.
Aby bol takýto útok možný a prakticky realizovateľný, priestor prehľadávaných kľúčov nesmie byť vačší ako hranica daná dostupnými
prostriedkami alebo časom potrebným na riešenie.

Pre ilustráciu si uveďme jednoduchý príklad. Majme zašifrovaný text \textquote{VECDOXSORSCDYBSMUIMRCSPSOBXKQBSNO}, ktorý vieme že bol zašifrovaný šifrou podobnou Cézarovej šifre.
Pre získanie otvoreného textu potrebujeme vyskúšať všetkých 26 možností posunov, čo je v tomto prípade kľúč, tak aby sme dostali zmysluplný text.

\begin{verbatim}
klúč 1
VECDOXSORSCDYBSMUIMRCSPSOBXKQBSNO
WFDEPYTPSTDEZCTNVJNSDTQTPCYLRCTOP

klúč 2
VECDOXSORSCDYBSMUIMRCSPSOBXKQBSNO
XGEFQZUQTUEFADUOWKOTEURUQDZMSDUPQ

kľúč 3
VECDOXSORSCDYBSMUIMRCSPSOBXKQBSNO
YHFGRAVRUVFGBEVPXLPUFVSVREANTEVQR

... // ďaľšie klúče 4..26
\end{verbatim}

Po prezretí všetkých možností by sme zistili že kľúč 16 sa dešifruje na \textquote{LUSTENIEHISTORICKYCHSIFIERNAGRIDE}.

% \todo{praktickosť útoku}

\subsubsection{Slovníkový útok}
Slovníkový útok narozdiel od útoku hrubou silou skúša iba niektoré možnosti z vopred pripraveného slovníka kľúčov.

Ukážme si ako by v príncípe mohol fungovať slovníkový útok na šifru Vigenere.
Nech zašifrovaný text je \textquote{SYKESUMWSWZXGCWJOQNVZMXTSYRSRFPHW}. Útočník má k dispozícii slovník slov \textquote{ABC, SOMAR, HESLO, ...}.

\begin{verbatim}
kľúč JANO
SYKESUMWSWZXGCWJOQNVZMXTSYRSRFPHW
JYXQJUZIJWMJXCJVFQAHQMKFJYEEIFCTN

kľúč SOMAR
SYKESUMWSWZXGCWJOQNVZMXTSYRSRFPHW
AKYEBCYKSFHJUCFRAENEHYLTBGDGROXTK

klúč HESLO
SYKESUMWSWZXGCWJOQNVZMXTSYRSRFPHW
LUSTENIEHISTORICKYCHSIFIERNAGRIDE
\end{verbatim}

\subsubsection{Genetické a evolučné algoritmy}

% Grid/kluster hpc.stuba.sk
Jedným z cieľov práce je preskúmať možnosti aplikovania útokov na klasické šifry v gridovom prostredí.
Grid možeme chápať ako skupinu počítačov, uzlov, spojenú pomocou siete LAN, prípadne inou sieťovou technológiou,
ktoré môžu ale nemusia byť geograficky oddelené.
Účelom takýchto počítačov je poskytnúť veľký výpočtový výkon, ktorý je použitý na riešenie špecifických úloh.

% hpc.stuba.sk (popis, pouzivanie)
V rámci Slovenskej technickej univerzity (STU), Centra výpočtovej techniky (CVT) sa nachádza superpočítač IBM iDataPlex, ktorý pozostáva z 52 výpočtových uzlov.
Každý výpočtový uzol má nasledovnú konfiguráciu:
\begin{itemize}
\item \gls{CPU}: 2 x 6 jadrový Intel Xeon X5670 2.93 GHz
\item \gls{RAM}: 48GB (24GB na procesor)
\item \gls{HDD}: 2TB 7200 RPM SATA
\item \gls{GPU}: 2 x NVIDIA Tesla M2050 448 cuda jadier, 3GB ECC RAM
\item Operačný systém: Scientific Linux 6.4
\item Sieťové pripojenie: 2 x 10Gb/s Ethernet
\end{itemize}
Spolu máme k dispozícii 624 \gls{cpu}, 3584 cuda jadier, 2,5TB \gls{ram}, 104TB lokálneho úložného priestoru a daľších 115TB zdielaného úložiska.
Výpočtový výkon dosahuje 6,76 TFLOPS a maximálny príkon aj vrátane chladenia je 40kW ref{hpc.stuba.sk}.

Aby sme boli schopný grid používať musíme si najprv zaregistrovať projekt a požiadať o vytvorenie
používatelského účtu na stránke výpočtového strediska \url{hpc.stuba.sk}.
Po registracii a ziskaní prihlasovacích udajov sa môžeme prihlásiť do webového rozhrania, cez ktoré možeme spravovať projekt,
pridávať daľsích riešitelov, prezerať si štatistiky a grafy.
Dôležitou funkciou webového rozhrania je zmena hesla používateľa a pridanie SSH verejného klúča, pomocou ktorého sa možeme prihlasovať bez zadávania hesla.

Do gridu sa možeme prihlasiť cez \textit{ssh} zadaním príkazu \textquote{\textit{ssh pouzivatelske_meno@hpc.stuba.sk}} a následne zadaním hesla.
Ak sa pripájame mimo univerzitnej siete STU, na prihlásenie musíme použit VPN.

% podporovane technologie (MPI)
% Komunikacia procesov

% Navrh aplikacie, c++, mpi, boost, cmake, serializacia
%% Geneticke algoritmy
%% Schemy GA
%% Vysledky GA
\begin{figure}[!ht]
\def\svgwidth{\columnwidth}
\centering
\begin{gnuplot}[terminal=pdf,terminaloptions=color]
set terminal pdf enhanced size 16cm, 10cm
set xrange [0:2000]
set yrange [0:1]

set key samplen 3 spacing 1 font ',10' left title 'Schema'

set xlabel "Počet znakov zašifrovaného textu"
set ylabel "Úspešnosť schémy (%)"

plot "data/10000_10_A" smooth csplines dt 1 title 'A', \
     "data/10000_10_B" smooth csplines dt 2 title 'B', \
     "data/10000_10_C" smooth csplines dt 3 title 'C', \
     "data/10000_10_D" smooth csplines dt 4 title 'D', \
     "data/10000_10_E" smooth csplines dt 5 title 'E', \
     "data/10000_10_F" smooth csplines dt 1 title 'F', \
     "data/10000_10_G" smooth csplines dt 2 title 'G', \
     "data/10000_10_H" smooth csplines dt 3 title 'H', \
     "data/10000_10_I" smooth csplines dt 4 title 'I', \
     "data/10000_10_J" smooth csplines dt 5 title 'J'

\end{gnuplot}
\caption{Počet iterácii: 10000, počiatočná populácia: 10}
\label{schema:ga_10000_10}
\end{figure}

\begin{figure}[!ht]
\centering
\begin{gnuplot}[terminal=pdf,terminaloptions=color]
set terminal pdf enhanced size 15cm, 8cm
set xrange [0:2000]
set yrange [0:1]

set key samplen 3 spacing 1 font ',10' left title 'Schema'

set xlabel "Počet znakov zašifrovaného textu"
set ylabel "Úspešnosť schémy (%)"

plot "data/10000_20_A" smooth csplines dt 1 title 'A', \
     "data/10000_20_B" smooth csplines dt 2 title 'B', \
     "data/10000_20_C" smooth csplines dt 3 title 'C', \
     "data/10000_20_D" smooth csplines dt 4 title 'D', \
     "data/10000_20_E" smooth csplines dt 5 title 'E', \
     "data/10000_20_F" smooth csplines dt 1 title 'F', \
     "data/10000_20_G" smooth csplines dt 2 title 'G', \
     "data/10000_20_H" smooth csplines dt 3 title 'H', \
     "data/10000_20_I" smooth csplines dt 4 title 'I', \
     "data/10000_20_J" smooth csplines dt 5 title 'J'

\end{gnuplot}
\caption{Počet iterácii: 10000, počiatočná populácia: 20}
\label{schema:ga_10000_20}
\end{figure}

\begin{figure}[!htbp]
\def\svgwidth{\columnwidth}
\centering
\begin{gnuplot}[terminal=pdf,terminaloptions=color]
set terminal pdf enhanced size 15cm, 9cm
set xrange [0:2000]
set yrange [0:1]

set key samplen 3 spacing 1 font ',10' left title 'Schéma'

set xlabel "Počet znakov zašifrovaného textu"
set ylabel "Úspešnosť schémy (%)"

plot "data/ga/10000_50_A" smooth csplines dt 1 title 'A', \
     "data/ga/10000_50_B" smooth csplines dt 2 title 'B', \
     "data/ga/10000_50_C" smooth csplines dt 3 title 'C', \
     "data/ga/10000_50_D" smooth csplines dt 4 title 'D', \
     "data/ga/10000_50_E" smooth csplines dt 5 title 'E', \
     "data/ga/10000_50_F" smooth csplines dt 1 title 'F', \
     "data/ga/10000_50_G" smooth csplines dt 2 title 'G', \
     "data/ga/10000_50_H" smooth csplines dt 3 title 'H', \
     "data/ga/10000_50_I" smooth csplines dt 4 title 'I', \
     "data/ga/10000_50_J" smooth csplines dt 5 title 'J'

\end{gnuplot}
\caption{Závislosť úspešnosti lúštenia od dĺžky ZT (10000 iterácií, 50 jedincov)}
\label{schema:ga_10000_50}
\end{figure}

\begin{figure}[!htbp]
\def\svgwidth{\columnwidth}
\centering
\begin{gnuplot}[terminal=pdf,terminaloptions=color]
set terminal pdf enhanced size 15cm, 9cm
set xrange [0:2000]
set yrange [0:1]

set key samplen 3 spacing 1 font ',10' left title 'Schéma'

set xlabel "Počet znakov zašifrovaného textu"
set ylabel "Úspešnosť schémy (%)"

plot "data/ga/10000_100_A" smooth csplines dt 1 title 'A', \
     "data/ga/10000_100_B" smooth csplines dt 2 title 'B', \
     "data/ga/10000_100_C" smooth csplines dt 3 title 'C', \
     "data/ga/10000_100_D" smooth csplines dt 4 title 'D', \
     "data/ga/10000_100_E" smooth csplines dt 5 title 'E', \
     "data/ga/10000_100_F" smooth csplines dt 1 title 'F', \
     "data/ga/10000_100_G" smooth csplines dt 2 title 'G', \
     "data/ga/10000_100_H" smooth csplines dt 3 title 'H', \
     "data/ga/10000_100_I" smooth csplines dt 4 title 'I', \
     "data/ga/10000_100_J" smooth csplines dt 5 title 'J'

\end{gnuplot}
\caption{Závislosť úspešnosti lúštenia od dĺžky ZT (10000 iterácii, 100 jedincov)}
\label{schema:ga_10000_100}
\end{figure}

\begin{figure}
\centering
\begin{gnuplot}[terminal=pdf,terminaloptions=color]
set terminal pdf enhanced size 14.5cm, 12cm
set xrange [0:2000]
set yrange [0:1]

set key samplen 3 spacing 1 font ',10' left

set xlabel "Počet znakov zašifrovaného textu"
set ylabel "Úspešnosť schémy (%)"

plot "data/50000_10_A" smooth csplines title 'A', \
     "data/50000_10_B" smooth csplines title 'B', \
     "data/50000_10_C" smooth csplines title 'C', \
     "data/50000_10_D" smooth csplines title 'D', \
     "data/50000_10_E" smooth csplines title 'E', \
     "data/50000_10_F" smooth csplines title 'F', \
     "data/50000_10_G" smooth csplines title 'G', \
     "data/50000_10_H" smooth csplines title 'H', \
     "data/50000_10_I" smooth csplines title 'I', \
     "data/50000_10_J" smooth csplines title 'J'

\end{gnuplot}
\caption{Počet iterácii: 50000, počiatočná populácia: 10}
\end{figure}

\begin{figure}
\centering
\begin{gnuplot}[terminal=pdf,terminaloptions=color]
set terminal pdf enhanced size 14.5cm, 12cm
set xrange [0:2000]
set yrange [0:1]

set key samplen 3 spacing 1 font ',10' left title 'Schema'

set xlabel "Počet znakov zašifrovaného textu"
set ylabel "Úspešnosť schémy (%)"

plot "data/50000_20_A" smooth csplines dt 1 title 'A', \
     "data/50000_20_B" smooth csplines dt 2 title 'B', \
     "data/50000_20_C" smooth csplines dt 3 title 'C', \
     "data/50000_20_D" smooth csplines dt 4 title 'D', \
     "data/50000_20_E" smooth csplines dt 5 title 'E', \
     "data/50000_20_F" smooth csplines dt 1 title 'F', \
     "data/50000_20_G" smooth csplines dt 2 title 'G', \
     "data/50000_20_H" smooth csplines dt 3 title 'H', \
     "data/50000_20_I" smooth csplines dt 4 title 'I', \
     "data/50000_20_J" smooth csplines dt 5 title 'J'

\end{gnuplot}
\caption{Počet iterácii: 50000, počiatočná populácia: 20}
\end{figure}

\begin{figure}[!htbp]
\def\svgwidth{\columnwidth}
\centering
\begin{gnuplot}[terminal=pdf,terminaloptions=color]
set terminal pdf enhanced size 15cm, 9cm
set xrange [0:2000]
set yrange [0:1]

set key samplen 3 spacing 1 font ',10' left title 'Schéma'

set xlabel "Počet znakov zašifrovaného textu"
set ylabel "Úspešnosť schémy (%)"

plot "data/ga/50000_50_A" smooth csplines dt 1 title 'A', \
     "data/ga/50000_50_B" smooth csplines dt 2 title 'B', \
     "data/ga/50000_50_C" smooth csplines dt 3 title 'C', \
     "data/ga/50000_50_D" smooth csplines dt 4 title 'D', \
     "data/ga/50000_50_E" smooth csplines dt 5 title 'E', \
     "data/ga/50000_50_F" smooth csplines dt 1 title 'F', \
     "data/ga/50000_50_G" smooth csplines dt 2 title 'G', \
     "data/ga/50000_50_H" smooth csplines dt 3 title 'H', \
     "data/ga/50000_50_I" smooth csplines dt 4 title 'I', \
     "data/ga/50000_50_J" smooth csplines dt 5 title 'J'

\end{gnuplot}
\caption{Závislosť úspešnosti lúštenia od dĺžky ZT (50000 iterácií, 50 jedincov)}
\label{schema:ga_50000_50}
\end{figure}

\begin{figure}[!ht]
\def\svgwidth{\columnwidth}
\centering
\begin{gnuplot}[terminal=pdf,terminaloptions=color]
set terminal pdf enhanced size 16cm, 10cm
set xrange [0:2000]
set yrange [0:1]

set key samplen 3 spacing 1 font ',10' left title 'Schema'

set xlabel "Počet znakov zašifrovaného textu"
set ylabel "Úspešnosť schémy (%)"

plot "data/50000_100_A" smooth csplines dt 1 title 'A', \
     "data/50000_100_B" smooth csplines dt 2 title 'B', \
     "data/50000_100_C" smooth csplines dt 3 title 'C', \
     "data/50000_100_D" smooth csplines dt 4 title 'D', \
     "data/50000_100_E" smooth csplines dt 5 title 'E', \
     "data/50000_100_F" smooth csplines dt 1 title 'F', \
     "data/50000_100_G" smooth csplines dt 2 title 'G', \
     "data/50000_100_H" smooth csplines dt 3 title 'H', \
     "data/50000_100_I" smooth csplines dt 4 title 'I', \
     "data/50000_100_J" smooth csplines dt 5 title 'J'

\end{gnuplot}
\caption{Počet iterácii: 50000, počiatočná populácia: 100}
\label{schema:ga_50000_100}
\end{figure}


%% Paralelene Geneticke algoritmy
%% Trieda Migrator
%% Schemy PGA
%% Vysledky PGA
%% Spolupraca s Petrom Javorkom
%% Spotrebovany cas/peniaze