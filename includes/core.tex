\section{Klasické šifry}
V tejto kapitole sa budeme zaoberať históriou a stručným prehľadom klasických šifier.
Spomenieme si aj niektoré základné útoky na klasické šifry. 

\subsection{História}
História klasických šifier a utajovania písomného textu je pravdepodobne tak stará ako samotné písmo.
Písmo, v podobe akej ho poznáme a používame dnes, pravdepodobne pochádza asi spred 3000 rokov pred Kristom a za jeho objaviteľov sa považujú
Feničania.
V niektorých prípadoch predstavovalo už použitie písma utajenie samotného textu.
Príkladom môžu byť Egyptské hieroglyfy alebo klinové písmo používané v Mezopotámii.
Iným príkladom môžu byť semitské jazyky, ktoré sú charakteristické používaním iba spoluhlások bez použitia samohlások,
pretože tie zaviedli až Aremejci a po nich následné Gréci aby pomocou nich boli schopný rozlíšiť jazyky \cite{ks}.
Aj diakritika ako taká má schopnosť rozlišovať významy slov, čo si ale až do 15.storočia nikto nevšímal,
až pokiaľ ju Arabi nezačali používať pri kryptoanalýze rôznych šifier.

Z historického hľadiska nie je možné presne zoradiť ako jednotlivé šifry vznikali, pretože súčasne vznikali na viacerých miestach sveta.
Komunikácia a s ňou spojené sírenie informácii nebolo také rýchle ako dnes, až do roku 1440 keď Johan Guttenberg vynašiel kníhtlač,
čo zjednodušilo výmenu a uchovávanie informácii. (TODO: pridať utajovanie informácie)

Ku kryptografii ako aj k rôznym iným vedným disciplínam prispelo v minulosti staré Grécko.
Jedným z najvýznamnejších príspevkov starých Grékov bolo široké rozšírenie abecedy a písomného prejavu.
Gréci písmo prebrali od Feničanov, ktorí na rozdiel od Egypťanov používali jednoduchšie písmo.

V Európe vďaka rozšíreniu abecedy začali vznikať aj prvé šifry, medzi ktoré patrí napríklad Cézarova šifra, ktorá vznikla v Rímskej ríši.
Iným príkladom môže byť transpozičná šifra skytalé, ktorá bola používaná v Sparte.

Pád Rímskej ríše spôsobil úpadok kryptografie, ktorý trval až do obdobia stredoveku. Typickým znakom kryptografie v tomto období bolo
napríklad písanie odzadu, alebo vertikálne, používanie cudzích jazykov, alebo vynechávanie samohlások \cite{ks}.

V stredoveku, kvôli bojom medzi pápežmi Ríma a Avignonu, bola kryptografia zdokonalená a začali sa používať rôzne kódy a nomenklátory.
Ich charakteristickým znakom bolo zamieňanie písmen alebo nahradzovanie mien a titulov osôb v správach.
V tomto období zabezpečovanie utajenia správ pokročilo až na takú úroveň, že na doručovanie správ boli použitý špeciálne vycvičení kuriéri.

V prvej polovici 20. storočia ľudia, ktorí pracovali v oblasti utajovanej komunikácie verili, že na to aby bola zabezpečená utajovaná komunikácia musí byť utajený kľúč a okrem neho aj šifrovací algoritmus. Toto ale odporovalo Kerckhoffovmu princípu, ktorý hovorí že: \textquote{Bezpečnosť šifrovacieho algoritmu musí závisieť výlučne na utajení kľúča a nie algoritmu}. Okrem toho sformuloval aj niekoľko požiadaviek na kryptografický systém, medzi ktoré patria:
\begin{enumerate}
\item systém musí byť teoreticky, alebo aspoň prakticky bezpečný
\item narušenie systému nesmie priniesť ťažkosti odosielateľovi a adresátovi
\item kľúč musí byť ľahko zapamätateľný a ľahko vymeniteľný
\item zašifrovaná správa musí byť prenášateľná telegrafom
\item šifrovacia pomôcka musí byť ľahko prenosná a ovládateľná jedinou osobou
\item systém musí byť jednoduchý, bez dlhého zoznamu pravidiel, nevyžadujúci nadmerné sústredenie
\end{enumerate}
Tieto princípy sú popísané v pôvodnej publikácii od Kerckhoffa \cite{kerckhoff}.

Existovala ale aj iná skupina vedcov, medzi ktorých patril aj Lester S. Hill, ktorý si uvedomoval že kryptológia je úzko spätá z matematikou.
V roku 1941 si na Hillových prácach zakladal A. Adrian Albert, ktorý pochopil, že v šifrovaní je možné použiť viacero algebraických štruktúr.
Neskôr toto všetko usporiadal a zdokonalil Claude E. Shannon, čo možno považovať za ukončenie éry klasických šifier \cite{ks}.

\todo{Možno pridať/spomenúť steganografiu.}

\subsection{Charakteristika}
Na rozdiel od moderných šifier, ktoré sa používajú dnes, sú tie klasické rozdielne v niektorých hlavných črtách.
Môžeme spomenúť niekoľko:
\begin{itemize}
\item Šifrovanie a dešifrovanie klasickej šifry možno realizovať zväčša pomocou papiera a ceruzky alebo nejakej mechanickej pomôcky.
\item V dnešnej dobe aj vďaka rozšírenému použitiu počítačov stratila väčšina týchto algoritmov svoj význam.
\item Utajuje sa algoritmus a aj kľúč a neuplatňuje sa Kerckhoffov princíp.
\item Na rozdiel od moderných šifier sa používajú malé abecedy.
\item V klasických šifrách je otvorený text, zašifrovaný text a kľúč v abecede reálneho jazyka, pričom v moderných šifrách sa používa binárne kódovanie.
\item Na klasické šifry sa zväčša dá použiť štatistická analýza. 
\end{itemize}
Z spomenutých charakteristík existujú aj výnimky. Napríklad pri Vigenerovej šifre sa algoritmus neutajoval. To platí aj pre Vernamovu šifru, ktorá okrem toho používa navyše binárne znaky. Vernamova šifra je perfektne bezpečná v podľa Shannonovej teórie \cite{ks}.

Klasické šifry môžeme rozdeliť do niekoľkých základných kategórii:
\begin{itemize}
\item Substitučné šifry
\item Transpozičné šifry
\item Homofónne šifry
\item Substitučno-permutačné šifry
\end{itemize}

\subsection{Útoky}
\todo{ks 2.3}
\subsubsection{Hrubou silou}
Útok hrubou silou (bruteforce) je typ útoku, ktorý sa snaží zlomiť kľúč vyskúšaním každej možnej kombinácie znakov z abecedy.

Pre ilustráciu si uveďme jednoduchý príklad. Máme zašifrovaný text \textquote{VECDOXSORSCDYBSMUIMRCSPSOBXKQBSNO}, ktorý vieme že bol zašifrovaný šifrou podobnou Cézarovej šifre.
Pre získanie otvoreného textu potrebujeme vyskúšať všetkých 26 možností posunov, čo je v tomto prípade kľúč, tak aby sme dostali zmysluplný text.

\begin{verbatim}
klúč 1
VECDOXSORSCDYBSMUIMRCSPSOBXKQBSNO
WFDEPYTPSTDEZCTNVJNSDTQTPCYLRCTOP

klúč 2
VECDOXSORSCDYBSMUIMRCSPSOBXKQBSNO
XGEFQZUQTUEFADUOWKOTEURUQDZMSDUPQ

kľúč 3
VECDOXSORSCDYBSMUIMRCSPSOBXKQBSNO
YHFGRAVRUVFGBEVPXLPUFVSVREANTEVQR

...
\end{verbatim}

Po prezretí všetkých možností by sme zistili že kľúč 16 sa dešifruje na \textquote{LUSTENIEHISTORICKYCHSIFIERNAGRIDE}.

\todo{praktickosť útoku}

\subsection{Slovníkový útok}
Slovníkový útok narozdiel od útoku hrubou silou skúša iba niektoré možnosti z vopred pripraveného slovníka kľúčov.