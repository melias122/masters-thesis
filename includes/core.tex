\section{Klasické šifry}
V tejto kapitole sa budeme zaoberať históriou a stručným prehľadom klasických šifier.
Spomenieme si aj niektoré základné útoky na klasické šifry. 

\subsection{História}
História klasických šifier a utajovania písomného textu je pravdepodobne tak stará ako samotné písmo.
Písmo, v podobe akej ho poznáme a používame dnes, pravdepodobne pochádza asi spred 3000 rokov pred Kristom a za jeho objaviteľov sa považujú
Feničania.
V niektorých prípadoch predstavovalo už použitie písma utajenie samotného textu.
Príkladom môžu byť Egyptské hieroglyfy alebo klinové písmo používané v Mezopotámii.
Iným príkladom môžu byť semitské jazyky, ktoré sú charakteristické používaním iba spoluhlások bez použitia samohlások,
pretože tie zaviedli až Aremejci a po nich následné Gréci aby pomocou nich boli schopný rozlíšiť jazyky \cite{ks}.
Aj diakritika ako taká má schopnosť rozlišovať významy slov, čo si ale až do 15.storočia nikto nevšímal,
až pokiaľ ju Arabi nezačali používať pri kryptoanalýze rôznych šifier.

Z historického hľadiska nie je možné presne zoradiť ako jednotlivé šifry vznikali, pretože súčasne vznikali na viacerých miestach sveta.
Komunikácia a s ňou spojené sírenie informácii nebolo také rýchle ako dnes, až do roku 1440 keď Johan Guttenberg vynašiel kníhtlač,
čo zjednodušilo výmenu a uchovávanie informácii.
% (TODO: pridať utajovanie informácie)

Ku kryptografii ako aj k rôznym iným vedným disciplínam prispelo v minulosti staré Grécko.
Jedným z najvýznamnejších príspevkov starých Grékov bolo široké rozšírenie abecedy a písomného prejavu.
Gréci písmo prebrali od Feničanov, ktorí na rozdiel od Egypťanov používali jednoduchšie písmo.

V Európe vďaka rozšíreniu abecedy začali vznikať aj prvé šifry, medzi ktoré patrí napríklad Cézarova šifra, ktorá vznikla v Rímskej ríši.
Iným príkladom môže byť transpozičná šifra skytalé, ktorá bola používaná v Sparte.

Pád Rímskej ríše spôsobil úpadok kryptografie, ktorý trval až do obdobia stredoveku. Typickým znakom kryptografie v tomto období bolo
napríklad písanie odzadu, alebo vertikálne, používanie cudzích jazykov, alebo vynechávanie samohlások \cite{ks}.

V stredoveku, kvôli bojom medzi pápežmi Ríma a Avignonu, bola kryptografia zdokonalená a začali sa používať rôzne kódy a nomenklátory.
Ich charakteristickým znakom bolo zamieňanie písmen alebo nahradzovanie mien a titulov osôb v správach.
V tomto období zabezpečovanie utajenia správ pokročilo až na takú úroveň, že na doručovanie správ boli použitý špeciálne vycvičení kuriéri.

V prvej polovici 20. storočia ľudia, ktorí pracovali v oblasti utajovanej komunikácie verili, že na to aby bola zabezpečená utajovaná komunikácia musí byť utajený kľúč a okrem neho aj šifrovací algoritmus. Toto ale odporovalo Kerckhoffovmu princípu, ktorý hovorí že: \textquote{Bezpečnosť šifrovacieho algoritmu musí závisieť výlučne na utajení kľúča a nie algoritmu}. Okrem toho sformuloval aj niekoľko požiadaviek na kryptografický systém, medzi ktoré patria:
\begin{enumerate}
\item systém musí byť teoreticky, alebo aspoň prakticky bezpečný
\item narušenie systému nesmie priniesť ťažkosti odosielateľovi a adresátovi
\item kľúč musí byť ľahko zapamätateľný a ľahko vymeniteľný
\item zašifrovaná správa musí byť prenášateľná telegrafom
\item šifrovacia pomôcka musí byť ľahko prenosná a ovládateľná jedinou osobou
\item systém musí byť jednoduchý, bez dlhého zoznamu pravidiel, nevyžadujúci nadmerné sústredenie
\end{enumerate}
Tieto princípy sú popísané v pôvodnej publikácii od Kerckhoffa \cite{kerckhoff}.

Existovala ale aj iná skupina vedcov, medzi ktorých patril aj Lester S. Hill, ktorý si uvedomoval že kryptológia je úzko spätá z matematikou.
V roku 1941 si na Hillových prácach zakladal A. Adrian Albert, ktorý pochopil, že v šifrovaní je možné použiť viacero algebraických štruktúr.
Neskôr toto všetko usporiadal a zdokonalil Claude E. Shannon, čo možno považovať za ukončenie éry klasických šifier \cite{ks}.

% \todo{Možno pridať/spomenúť steganografiu.}

\subsection{Charakteristika}
Na rozdiel od moderných šifier, ktoré sa používajú dnes, sú tie klasické rozdielne v niektorých hlavných črtách.
Môžeme spomenúť niekoľko:
\begin{itemize}
\item Šifrovanie a dešifrovanie klasickej šifry možno realizovať zväčša pomocou papiera a ceruzky alebo nejakej mechanickej pomôcky.
\item V dnešnej dobe aj vďaka rozšírenému použitiu počítačov stratila väčšina týchto algoritmov svoj význam.
\item Utajuje sa algoritmus a aj kľúč a neuplatňuje sa Kerckhoffov princíp.
\item Na rozdiel od moderných šifier sa používajú malé abecedy.
\item V klasických šifrách je otvorený text, zašifrovaný text a kľúč v abecede reálneho jazyka, pričom v moderných šifrách sa používa binárne kódovanie.
\item Na klasické šifry sa zväčša dá použiť štatistická analýza. 
\end{itemize}
Z spomenutých charakteristík existujú aj výnimky. Napríklad pri Vigenerovej šifre sa algoritmus neutajoval. To platí aj pre Vernamovu šifru, ktorá okrem toho používa navyše binárne znaky. Vernamova šifra je perfektne bezpečná v podľa Shannonovej teórie \cite{ks}.

Klasické šifry môžeme rozdeliť do niekoľkých základných kategórii:
\begin{itemize}

\item \textbf{Substitučné šifry.}
  V prípade že šifra permutuje znaky zdrojovej abecedy, hovoríme o monoalfabetickej šifre.
  Ako príklad môžeme uviesť šifru Atbaš prípadne Cézarovu šifru, alebo iné.
  V inom prípade ak sa aplikuje viacero permutácii podľa polohy znaku v otvorenom texte, tak hovoríme o polyalfabetickej šifre.
  Príkladom je Vigenerova šifra. Ďalším prípadom je polygramová šifra, kde sa z otvoreného textu najprv vytvoria bloky,
  na ktoré sa potom aplikuje nejaká permutácia.

\item \textbf{Transpozičné šifry.}
  Transpozičné šifry sú vlastne blokové šifry, ktoré pri šifrovaní a dešifrovaní aplikujú pevne zvolenú permutáciu na každý blok
  otvoreného/zašifrovaného textu. Od polyalfabetickej šifry sa líši v poradí vykonávania operácii.
  
\item \textbf{Homofónne šifry.}
  Homofónne šifry sú šifry, ktoré majú znáhodnený zašifrovaný text. Tieto šifry sa snažia zabrániť frekvenčnej analýze textu. 
  
\item \textbf{Substitučno-permutačné šifry.}
  Ak aplikujeme viacero substitučný a permutačných šifier na otvorený text tak hovoríme o substitučno-permutačných šifrách.
  Šifrovanie prebieha tak, že blok otvoreného textu sa rozdelí na menšie bloky, na ktoré je potom aplikovaná substitúcia, a permutácia,
  ktorá sa aplikuje na celý blok. Substitúcia zabezpečuje konfúziu a permutácia difúziu.
  
\end{itemize}

\subsection{Útoky}
% \todo{ks 2.3}
\subsubsection{Hrubou silou}
Útok hrubou silou (bruteforce) je typ útoku, ktorý sa snaží zlomiť kľúč tak, že sa prehľadáva celý priestor kľúčov.
Aby bol takýto útok možný a prakticky realizovateľný, priestor prehľadávaných kľúčov nesmie byť väčší ako hranica daná dostupnými
prostriedkami alebo časom potrebným na riešenie.

Pre ilustráciu si uveďme jednoduchý príklad. Majme zašifrovaný text \textquote{VECDOXSORSCDYBSMUIMRCSPSOBXKQBSNO}, ktorý vieme že bol zašifrovaný šifrou podobnou Cézarovej šifre.
Pre získanie otvoreného textu potrebujeme vyskúšať všetkých 26 možností posunov, čo je v tomto prípade kľúč, tak aby sme dostali zmysluplný text.

\begin{verbatim}
klúč 1
VECDOXSORSCDYBSMUIMRCSPSOBXKQBSNO
WFDEPYTPSTDEZCTNVJNSDTQTPCYLRCTOP

klúč 2
VECDOXSORSCDYBSMUIMRCSPSOBXKQBSNO
XGEFQZUQTUEFADUOWKOTEURUQDZMSDUPQ

kľúč 3
VECDOXSORSCDYBSMUIMRCSPSOBXKQBSNO
YHFGRAVRUVFGBEVPXLPUFVSVREANTEVQR

... // ďaľšie klúče 4..26
\end{verbatim}

Po prezretí všetkých možností by sme zistili že kľúč 16 sa dešifruje na \textquote{LUSTENIEHISTORICKYCHSIFIERNAGRIDE}.

% \todo{praktickosť útoku}

\subsubsection{Slovníkový útok}
Slovníkový útok narozdiel od útoku hrubou silou skúša iba niektoré možnosti z vopred pripraveného slovníka kľúčov.

Ukážme si ako by v princípe mohol fungovať slovníkový útok na šifru Vigenere.
Nech zašifrovaný text je \textquote{SYKESUMWSWZXGCWJOQNVZMXTSYRSRFPHW}. Útočník má k dispozícii slovník slov \textquote{ABC, SOMAR, HESLO, ...}.

\begin{verbatim}
kľúč JANO
SYKESUMWSWZXGCWJOQNVZMXTSYRSRFPHW
JYXQJUZIJWMJXCJVFQAHQMKFJYEEIFCTN

kľúč SOMAR
SYKESUMWSWZXGCWJOQNVZMXTSYRSRFPHW
AKYEBCYKSFHJUCFRAENEHYLTBGDGROXTK

klúč HESLO
SYKESUMWSWZXGCWJOQNVZMXTSYRSRFPHW
LUSTENIEHISTORICKYCHSIFIERNAGRIDE
\end{verbatim}

\subsubsection{Genetické a evolučné algoritmy}
todo

\section{Grid}
Jedným z cieľov práce je preskúmať možnosti aplikovania útokov na klasické šifry v gridovom prostredí.
Grid môžeme chápať ako skupinu počítačov, uzlov, spojenú pomocou siete \acrfull{lan}, prípadne inou sieťovou technológiou,
ktoré môžu ale nemusia byť geograficky oddelené.
Účelom takýchto počítačov je poskytnúť veľký výpočtový výkon, ktorý je použitý na riešenie špecifických úloh.

\subsection{hpc.stuba.sk}
V rámci Slovenskej technickej univerzity (STU), Centra výpočtovej techniky (CVT) sa nachádza superpočítač IBM iDataPlex, ktorý pozostáva z 52 výpočtových uzlov.
Každý výpočtový uzol má nasledovnú konfiguráciu:
\begin{itemize}
\item \acrshort{cpu}: 2 x 6 jadrový Intel Xeon X5670 2.93 GHz
\item \acrshort{ram}: 48GB (24GB na procesor)
\item \acrshort{hdd}: 2TB 7200 RPM SATA
\item \acrshort{gpu}: 2 x NVIDIA Tesla M2050 448 cuda jadier, 3GB ECC \acrshort{ram}
\item Operačný systém: Scientific Linux 6.4
\item Sieťové pripojenie: 2 x 10Gb/s Ethernet
\end{itemize}
Spolu máme k dispozícii 624 \acrshort{cpu}, 3584 cuda jadier, 2,5TB \acrshort{ram} , 104TB lokálneho úložného priestoru a ďalších 115TB zdielaného úložiska.
Výpočtový výkon dosahuje 6,76 TFLOPS a maximálny príkon aj spolu s chladením je 40kW \cite{hpc}.

V tabuľke \ref{tab:filesystem} môžeme vidieť dostupné diskové umiestnenia pre každého používateľa, prípadne úlohu.
Umiestnenie \texttt{/home\$USER} je domovským priečinkom každého používateľa.
Jedno z obmedzení tohoto umiestnenia je že môže obsahovať maximálne osemdesiaťtisíc súborov a priečinkov.
Taktiež ma značne obmezdenú kapacitu čo sa nemusí hodiť pre každý typ úlohy.
Ďaľším umiestnením, ktoré ma používateľ k dispozícii je \texttt{/work/\$USER}.
Toto umiestnenie nemá žiadne väčšie odmedzenia slúži ako zdieľaný disk pre výpočty.
Môžeme tu vytvárať ľubovolný počet súborov a priečinkov, avšak podľa \cite{hpc} by sa tento disk mal využívať hlavne na prenos objemnejších dát v blokoch väčších ako 16kB. Obe spomenuté umiestnenia sú sieťové disky \acrshort{gpfs}.
Posledným umiestnením je \texttt{/scratch/\$PBS\_JOBID} alebo tiež aj \texttt{\$TMPDIR} v prípade \acrshort{pbs} skriptu.
Tento priestor je unikátny pre každú úlohu a je vhodný na spracovanie veľkého počtu malých súborov.
V prípade použitia tohto umiestnenia si treba dať pozor na zmazanie dáť, ktoré sa mažú ihneď po skončení úlohy.

\begin{table}[!h]
\centering
\label{tab:filesystem}
\begin{tabular}{@{}lllll@{}}
\toprule
\textbf{Filesystem}   & \textbf{Zálohovanie} & \textbf{Mazanie} & \textbf{Kapacita} & \textbf{Obmedzenia} \\ \midrule
\texttt{/home/\$USER}          & áno                  & nie              & 32GB              & 80k inodes          \\
\texttt{/scratch/\$PBS\_JOBID} & nie                  & ihneď            & 1.6TB             & nie                 \\
\texttt{/work/\$USER}          & nie                  & áno              & 56TB              & nie                 \\ \bottomrule
\end{tabular}
\caption{Disky}
\end{table}

Aby sme boli schopný grid používať musíme si najprv zaregistrovať projekt a požiadať o vytvorenie
používateľského účtu na stránke výpočtového strediska \url{hpc.stuba.sk}.
Po registrácii a získaní prihlasovacích údajov sa môžeme prihlásiť do webového rozhrania, cez ktoré môžeme spravovať projekt,
pridávať Ďaľších riešiteľov, prezerať si štatistiky a grafy.
Dôležitou funkciou webového rozhrania je zmena hesla používateľa a pridanie \acrshort{ssh} verejného kľúča, pomocou ktorého sa môžeme prihlasovať bez zadávania hesla.

\subsection{Príkazy}
Do gridu sa môžeme prihlásiť cez \acrshort{ssh} zadaním príkazu \texttt{ssh login@hpc.stuba.sk} a následným zadaním hesla v prípade ak nepoužívame prihlasovanie pomocou verejného kľúča.
Ak sa pripájame mimo univerzitnej siete STU, na prihlásenie musíme použiť \acrshort{vpn}.
Po pripojení máme k dispozícií štandardnú linuxovú konzolu, ktorá ale obsahuje niekoľko špecifických príkazov pre daný grid.
Zaujímať nás budú príkazy: \texttt{module, qstat, qfree, qsub, qsig}.
Niektoré výstupy sú pre svoju obsiahlosť skrátené.

\subsubsection{module}
Príkaz \texttt{module} slúži na rýchle nastavenie ciest k vybraným knižniciam. Existujúce moduly môžeme vypísať pomocou \texttt{module avail}

\begin{lstlisting}[caption={module avail}]
  --------------------------- /apps/modulefiles ---------------------------
  abyss/1.3.7                 gaussian/g03                mvapich2/2.1
  ansys/15.0                  gaussian/g09                mvapich2/2.2
  cmake/2.8.10.2              gcc/4.7.4(default)          nwchem/6.1.1(default)
  cmake/3.1.0                 gcc/4.8.4                   nwchem/6.6
  cp2k/2.5.1                  gcc/4.9.3                   openblas/0.2.18
  cuda/6.5                    gcc/5.4                     openmpi/1.10.2
  devel                       gcc/6.3                     openmpi/1.10.4
  dirac/13.3                  gridMathematica/9.0         openmpi/1.10.5
  dirac/14                    intel/composer_xe_2011      openmpi/1.4.5
  esi/pamstamp                intel/composer_xe_2013      openmpi/1.6.5(default)
  esi/pamstamp-platform       intel/libs_2011             openmpi/1.6.5-int8
  esi/procast                 intel/libs_2013             openmpi/1.7.2
  esi/sysweld                 matlab/R2015b               openmpi/1.7.5
  fftw3/3.3.3                 molcas/8.0                  openmpi/1.8.8
  fftw3/3.3.5                 mvapich2/1.8a2              openmpi/1.8.8-int8
  fftw3/intel-3.3.3           mvapich2/1.9(default)       openmpi/2.1.0
  fluent/15.0.7               mvapich2/2.0                openmpi/intel-1.10.4
\end{lstlisting}

Pre načítanie modulov zadáme \texttt{module load modul1 modul2 ...}, aktuálne používané moduly zobrazíme pomocou
\texttt{module list} a odstrániť ich môžeme príkazom \texttt{module purge}.
Podrobnejšie voľby príkazu \texttt{module} sa môžeme dozvedieť z manuálových stránok.

\subsubsection{qstat}
Ďalším dôležitým príkazom je \texttt{qstat}, ktorý zobrazuje status aktuálne bežiacich úloh.
Detailnejší výpis o nami spustených úlohách môžeme vypísať cez \texttt{qstat -u \$USER} alebo \texttt{qstat -a}

\begin{lstlisting}[caption={qstat}]
  Job ID          Name             User          Time Use   S Queue
  --------------- ---------------- ------------- ---------  - --------
  114557.one      halogen          3xjakubecj    499:03:0   R parallel
  114640.one      JerMnchexFq5     3breza        218:35:9   R parallel
  114663.one      Job4             3xrasova      78:07:20   R parallel
  114668.one      run.opt          3antusek      674:08:1   R parallel
  114692.one      Job5             3xbuchab      43:39:43   R parallel
  114710.one      PGA              3xelias       226:46:1   R parallel
\end{lstlisting}

\begin{lstlisting}[caption={qstat -u 3xelias}]
  Job ID      Queue    Jobname       SessID   TSK  Time    S   Time
  ----------  -------- ------------  ------  ----- ------ --------- - 
  114710.one  parallel PGA             3418   96   120:00:00 R 19:08:38
  115265.one  parallel PGA_Mpi_3_b    24619    4   120:00:00 R 31:17:51
  115266.one  parallel PGA_Mpi_3_d    14748    4   120:00:00 R 31:17:51
  115267.one  parallel PGA_Mpi_3_e    14780    4   120:00:00 R 31:17:51
  115268.one  parallel PGA_Mpi_5_b    16429    6   120:00:00 R 31:17:50
  115269.one  parallel PGA_Mpi_5_d    16471    6   120:00:00 R 31:17:49
  115270.one  parallel PGA_Mpi_5_e    22492    6   120:00:00 R 31:15:43
  115271.one  parallel PGA_Mpi_5_f    22450    6   120:00:00 R 31:15:45
  115272.one  parallel PGA_Mpi_11_b    4254   12   120:00:00 R 31:12:39
  115273.one  parallel PGA_Mpi_11_d     --    12   120:00:00 Q    -- 
  115274.one  parallel PGA_Mpi_11_e    1647   12   120:00:00 R 31:12:08
  115275.one  parallel PGA_Mpi_11_f   22605   12   120:00:00 R 31:11:37
\end{lstlisting}

Posledný riadok tabuľky príkazu \texttt{qstat -u 3xelias} popisuje nami spustenú úlohu.
Dôležité sú pre nás predovšetkým stĺpce \texttt{Time}, \texttt{Job ID}.
Posledný stĺpec \texttt{Time} hovorí o tom ako dlho je už naša úloha spustená, druhý stĺpec \texttt{Time} nám deklaruje maximálny možný čas,
ktorý má úloha PGA vyhradený. Hodnoty zo stĺpca \texttt{Job ID} môžeme použiť do príkazu \texttt{qsig} pre vynútené ukončenie úlohy.

\subsubsection{qfree}
Ak si chceme zobraziť aktuálne vyťaženie gridu, môžeme tak urobiť príkazom \texttt{qfree}.

\begin{lstlisting}[caption={qfree}]
  CLUSTER STATE SUMMARY                   Local    GPFS Storage
  
  Core        1  ...  12  load  FreeMem  Scratch  Read   Write  State
  Node  Queue                    [GB]     [GB]   [MB/s] [MB/s]
  
  comp01  S  [ ] ... [ ]  0.00  44.44   0 (0.0%)  0.00   0.00   free
  comp02  T  [ ] ... [ ]  0.00  44.45   0 (0.0%)  0.00   0.00   free
  ...
  comp44  P  [O] ... [O]  12.0  44.41   0 (0.0%)  0.00   0.00   job-exclusive
  comp45  P  [O] ... [O]  12.0  44.40   0 (0.0%)  0.00   0.00   job-exclusive
  comp46  P  [O] ... [O]  12.0  44.41   0 (0.0%)  0.00   0.00   job-exclusive
  comp47  P  [O] ... [O]  12.0  44.41   0 (0.0%)  0.00   0.00   job-exclusive
  comp48  P  [X] ... [X]  1.25  22.76   0 (0.0%) 45.50   2.83   job-exclusive
  gpu1    G  [X] ... [X]  7.98  38.33   0 (0.0%) 42.13   0.92   job-exclusive
  gpu2    G  [ ] ... [ ]  0.00  44.45   0 (0.0%)  0.00   0.00   free
  gpu3    G  [ ] ... [ ]  0.00  44.45   0 (0.0%)  0.00   0.00   free
  gpu4    G  [ ] ... [ ]  0.00  44.45   0 (0.0%)  0.00   0.00   free
\end{lstlisting}

Prvý stĺpec popisuje názov výpočtového uzlu. Druhý stĺpec označuje druh fronty. S je pre sériové úlohy, \texttt{T} pre interaktívne úlohy,
podobne \texttt{P} pre paralelné výpočty a \texttt{G} pre grafické výpočty.
Stĺpce jeden až dvanásť označujú procesory \acrshort{cpu} respektíve \acrshort{gpu} pre grafické výpočty.
Zvyšné stĺpce ako už možno vyčítať z názvu popisujú celkové zaťaženie výpočtového uzla, volnú pamäť a využitie diskov.
Posledný stĺpec \texttt{State} popisuje stav uzlu. Uzol môže byť volný alebo na vyťažený ak vykonáva nejakú úlohu.
Procesory na ktorých prebieha výpočet našej úlohy sú označené ako \texttt{[O]}, zvyšné vyťažené \acrshort{cpu} sú označené ako \texttt{[X]},
naopak voľné \acrshort{cpu} sú označené medzerou \texttt{[ ]} a v prípade že uzol nie je dostupný budú \acrshort{cpu} označené ako \texttt{[-]}.

Posledný a najdôležitejší príkaz je \texttt{qsub}, ktorý slúži na zaradenie úloh, \acrshort{pbs} skriptov do výpočtovej fronty.

\subsection{Výpočtové fronty}
Aby sme boli schopný spustiť akúkoľvek výpočtovú úlohu na gride, potrebujeme k tomu \acrfull{pbs} súbor.
\acrshort{pbs} súbor je v skutočnosti iba jednoduchý textový súbor, ktorý definuje požiadavky na výpočtové zdroje a príkazy pre grid.

V tabuľke \ref{tab:fronty} sa nachádzajú všetky výpočtové fronty, ktoré sú dostupné na gride \url{hpc.stuba.sk}.
Fronta debug slúži na rýchle odľadenie úloh. Úlohy v tejto fronte majú vysokú prioritu preto sú spustené takmer okamžite.
Debug fronta je obmedzená na maximálne dve súčasne spustené úlohy. Fronta gpu je ďalším typom fronty pre úlohy, ktoré využívajú grafický akcelerátor.
Pre úlohy, ktoré využívajú \acrshort{mpi}, \acrshort{omp} a iné knižnice na paralelné programovanie slúži fronta parallel.
Úlohy takého typu musia použiť minimálne štyry a maximálne deväťdesiatšesť \acrshort{cpu}.
Poslednou výpočtovou frontou je serial, na ktorej môžeme spúštať jednoprocesorové úlohy. 

% todo: check mmlsquota
\begin{table}[!h]
\centering
\label{tab:fronty}
\begin{tabular}{@{}lccc@{}}
\toprule
\textbf{Názov fronty} & \multicolumn{1}{l}{\textbf{walltime (max)}} & \multicolumn{1}{l}{\textbf{nodes}} & \multicolumn{1}{l}{\textbf{ppn}} \\ \midrule
debug                 & 30 minút                                    & -                                  & -                                      \\
gpu                   & 24 hodín                                    & -                                  & -                                      \\
parallel              & 240 hodín                                   & 1 - 8                              & 4 - 12                                 \\
serial                & 240 hodín                                   & 1                                  & 1                                      \\ \bottomrule
\end{tabular}
\caption{Výpočtové fronty a ich obmedzenia}
\end{table}

\subsection{Príklad sériovej úlohy}

\begin{lstlisting}[language=bash, caption={uloha1.pbs}, label={pbs:1}, numbers=left]
#!/bin/bash

#PBS -N uloha1
#PBS -l nodes=1:ppn=1
#PBS -l walltime=00:01:00  
#PBS -A 3ANTAL-2016
#PBS -q serial

cd /work/3xelias/uloha1
./seriova_uloha
\end{lstlisting}

Vo výpise \ref{pbs:1} môžeme vidieť príklad jednoduchého skriptu ktorý sériovej úlohy.
Popíšme si jednotlivé riadky skriptu:
\begin{itemize}
\item
  Prvý riadok v súbore definuje aký shell sa má použiť pre spustenie skriptu.
  V našom prípade sme použili \acrshort{bash}, ale mohli by sme použiť aj iný shell alebo skriptovací jazyk.
\item
  Tretí riadok určuje názov úlohy.
\item
  Štvrtý riadok vymedzuje koľko uzlov a procesorov si žiadame od gridu.
  V tomto prípade si žiadame jeden výpočtový uzol a jeden procesor.
\item
  Piaty riadok v \acrshort{pbs} skripte vymedzuje aké časové rozpätie potrebujeme pre úlohu.
  V tomto príklade si žiadame jednu minútu.
\item
  V šiestom riadku sa nachádza identifikátor podľa ktorého sa identifikujú úlohy s jednotlivými projektami.
  Tento parameter je povinný a možno ho získať po prihlásení na webový portál \url{https://www.hpc.stuba.sk/index.php?l=sk&page=login}.
\item
  Parameter -q v siedmom riadku definuje typ výpočtovej fronty do akej bude úloha zaradená.
  V tomto príklade chceme úlohu zaradiť do fronty \textquote{serial}.
\item
  V deviatom riadku sa presunieme do priečinku v ktorom sú uložené všetky potrebné dáta pre túto úlohu vrátane programu \textquote{seriova\_uloha}.
\item
  Posledný riadok spustí program \textquote{seriova\_uloha}.
\end{itemize}
Úlohu môže zaradiť do fronty príkazom \texttt{qsub uloha1.pbs}.

\subsection{Príklad paralelnej úlohy}
\begin{lstlisting}[language=bash, caption={uloha2.pbs}, label={pbs:2}, numbers=left]
#!/bin/bash

#PBS -N paralelna_uloha
#PBS -l nodes=5:ppn=12
#PBS -l walltime=48:00:00  
#PBS -A 3ANTAL-2016
#PBS -q parallel
#PBS -m ea
#PBS -M xelias@stuba.sk

. /etc/profile.d/modules.sh
module purge
module load gcc/5.4 openmpi/1.10.2

cd /work/3xelias/parallel
mpirun ./parallel
\end{lstlisting}

Podobne ako v predchádzajúcom príklade sériovej úlohy si popíšeme niektoré riadky príkladu \texttt{uloha2.pbs}:
\begin{itemize}
\item
  Na rozdiel od predchádzajúcej úlohy si v tomto príklade na riadku číslo štyri žiadame päť výpočtových uzlov a na každom uzle dvanásť \acrshort{cpu}.
  Dokopy si žiadame šesťdesiat procesorov.
\item
  V piatom riadku požadujeme časové rozpätie štyridsitich ôsmich hodín.
\item
  Siedmy riadok definuje paralelnú frontu.
\item
  Parametre \texttt{e} a \texttt{a} na riadku osem hovoria kedy sa má poslať email o zmene stavu úlohy.
  Parameter \texttt{e} znamená po skončení úlohy.
  Parameter \texttt{a} znamená pri zrušení úlohy.
  Ďalšie parametre môžu byť \texttt{b} (štart úlohy) a \texttt{n} (neposielať žiadny e-mail).
\item
  V deviatom riadku definujeme e-mailovú adresu, na ktorú bude zaslaný mail v prípade ak úloha skončí alebo bude prerušená.
\item
  V jedenástom riadku načítame všetky potrebné premenné prostredia pre moduly.
\item
  V dvanástom riadku odstránime všetky načítané moduly ak boli nejaké dostupné v premmenných prostredia.
\item
  V riadku číslo dvanásť tohto súboru načítame knižnicu gcc verzie 5.4 a knižnicu openmpi verzie 1.10.2.
  Pre správny beh programu by sa všetky načítané knižnice mali zhodovať stými, s ktorými bola aplikácia spúštaná v tomto skripte skompilovaná.
\item
  Podobne ako v ukázke \ref{pbs:1} sa prepneme do priečinku \texttt{/work/3xelias/parallel}, ktorý musí obsahovať vsetky potrebné dáta
  pre samotný program \texttt{parallel}.
\item
  Posledný riadok spustí program mpirun, ktorý potom spustí program parallel. Tento krok je vysvetlený v kapitole mpi. .Pridat odkaz.
\end{itemize}




% podporovane technologie (MPI)
% Komunikacia procesov

% Navrh aplikacie, poziadavky, c++, mpi, boost, cmake, serializacia, build.sh, run.sh
% \subsection{Genetické algoritmy}
%% Geneticke algoritmy
%% Schemy GA
%% Vysledky GA
%\begin{figure}[!ht]
\def\svgwidth{\columnwidth}
\centering
\begin{gnuplot}[terminal=pdf,terminaloptions=color]
set terminal pdf enhanced size 16cm, 10cm
set xrange [0:2000]
set yrange [0:1]

set key samplen 3 spacing 1 font ',10' left title 'Schema'

set xlabel "Počet znakov zašifrovaného textu"
set ylabel "Úspešnosť schémy (%)"

plot "data/10000_10_A" smooth csplines dt 1 title 'A', \
     "data/10000_10_B" smooth csplines dt 2 title 'B', \
     "data/10000_10_C" smooth csplines dt 3 title 'C', \
     "data/10000_10_D" smooth csplines dt 4 title 'D', \
     "data/10000_10_E" smooth csplines dt 5 title 'E', \
     "data/10000_10_F" smooth csplines dt 1 title 'F', \
     "data/10000_10_G" smooth csplines dt 2 title 'G', \
     "data/10000_10_H" smooth csplines dt 3 title 'H', \
     "data/10000_10_I" smooth csplines dt 4 title 'I', \
     "data/10000_10_J" smooth csplines dt 5 title 'J'

\end{gnuplot}
\caption{Počet iterácii: 10000, počiatočná populácia: 10}
\label{schema:ga_10000_10}
\end{figure}

%\begin{figure}[!ht]
\centering
\begin{gnuplot}[terminal=pdf,terminaloptions=color]
set terminal pdf enhanced size 15cm, 8cm
set xrange [0:2000]
set yrange [0:1]

set key samplen 3 spacing 1 font ',10' left title 'Schema'

set xlabel "Počet znakov zašifrovaného textu"
set ylabel "Úspešnosť schémy (%)"

plot "data/10000_20_A" smooth csplines dt 1 title 'A', \
     "data/10000_20_B" smooth csplines dt 2 title 'B', \
     "data/10000_20_C" smooth csplines dt 3 title 'C', \
     "data/10000_20_D" smooth csplines dt 4 title 'D', \
     "data/10000_20_E" smooth csplines dt 5 title 'E', \
     "data/10000_20_F" smooth csplines dt 1 title 'F', \
     "data/10000_20_G" smooth csplines dt 2 title 'G', \
     "data/10000_20_H" smooth csplines dt 3 title 'H', \
     "data/10000_20_I" smooth csplines dt 4 title 'I', \
     "data/10000_20_J" smooth csplines dt 5 title 'J'

\end{gnuplot}
\caption{Počet iterácii: 10000, počiatočná populácia: 20}
\label{schema:ga_10000_20}
\end{figure}

%\begin{figure}[!htbp]
\def\svgwidth{\columnwidth}
\centering
\begin{gnuplot}[terminal=pdf,terminaloptions=color]
set terminal pdf enhanced size 15cm, 9cm
set xrange [0:2000]
set yrange [0:1]

set key samplen 3 spacing 1 font ',10' left title 'Schéma'

set xlabel "Počet znakov zašifrovaného textu"
set ylabel "Úspešnosť schémy (%)"

plot "data/ga/10000_50_A" smooth csplines dt 1 title 'A', \
     "data/ga/10000_50_B" smooth csplines dt 2 title 'B', \
     "data/ga/10000_50_C" smooth csplines dt 3 title 'C', \
     "data/ga/10000_50_D" smooth csplines dt 4 title 'D', \
     "data/ga/10000_50_E" smooth csplines dt 5 title 'E', \
     "data/ga/10000_50_F" smooth csplines dt 1 title 'F', \
     "data/ga/10000_50_G" smooth csplines dt 2 title 'G', \
     "data/ga/10000_50_H" smooth csplines dt 3 title 'H', \
     "data/ga/10000_50_I" smooth csplines dt 4 title 'I', \
     "data/ga/10000_50_J" smooth csplines dt 5 title 'J'

\end{gnuplot}
\caption{Závislosť úspešnosti lúštenia od dĺžky ZT (10000 iterácií, 50 jedincov)}
\label{schema:ga_10000_50}
\end{figure}

%\begin{figure}[!htbp]
\def\svgwidth{\columnwidth}
\centering
\begin{gnuplot}[terminal=pdf,terminaloptions=color]
set terminal pdf enhanced size 15cm, 9cm
set xrange [0:2000]
set yrange [0:1]

set key samplen 3 spacing 1 font ',10' left title 'Schéma'

set xlabel "Počet znakov zašifrovaného textu"
set ylabel "Úspešnosť schémy (%)"

plot "data/ga/10000_100_A" smooth csplines dt 1 title 'A', \
     "data/ga/10000_100_B" smooth csplines dt 2 title 'B', \
     "data/ga/10000_100_C" smooth csplines dt 3 title 'C', \
     "data/ga/10000_100_D" smooth csplines dt 4 title 'D', \
     "data/ga/10000_100_E" smooth csplines dt 5 title 'E', \
     "data/ga/10000_100_F" smooth csplines dt 1 title 'F', \
     "data/ga/10000_100_G" smooth csplines dt 2 title 'G', \
     "data/ga/10000_100_H" smooth csplines dt 3 title 'H', \
     "data/ga/10000_100_I" smooth csplines dt 4 title 'I', \
     "data/ga/10000_100_J" smooth csplines dt 5 title 'J'

\end{gnuplot}
\caption{Závislosť úspešnosti lúštenia od dĺžky ZT (10000 iterácii, 100 jedincov)}
\label{schema:ga_10000_100}
\end{figure}

%\begin{figure}
\centering
\begin{gnuplot}[terminal=pdf,terminaloptions=color]
set terminal pdf enhanced size 14.5cm, 12cm
set xrange [0:2000]
set yrange [0:1]

set key samplen 3 spacing 1 font ',10' left

set xlabel "Počet znakov zašifrovaného textu"
set ylabel "Úspešnosť schémy (%)"

plot "data/50000_10_A" smooth csplines title 'A', \
     "data/50000_10_B" smooth csplines title 'B', \
     "data/50000_10_C" smooth csplines title 'C', \
     "data/50000_10_D" smooth csplines title 'D', \
     "data/50000_10_E" smooth csplines title 'E', \
     "data/50000_10_F" smooth csplines title 'F', \
     "data/50000_10_G" smooth csplines title 'G', \
     "data/50000_10_H" smooth csplines title 'H', \
     "data/50000_10_I" smooth csplines title 'I', \
     "data/50000_10_J" smooth csplines title 'J'

\end{gnuplot}
\caption{Počet iterácii: 50000, počiatočná populácia: 10}
\end{figure}

%\begin{figure}
\centering
\begin{gnuplot}[terminal=pdf,terminaloptions=color]
set terminal pdf enhanced size 14.5cm, 12cm
set xrange [0:2000]
set yrange [0:1]

set key samplen 3 spacing 1 font ',10' left title 'Schema'

set xlabel "Počet znakov zašifrovaného textu"
set ylabel "Úspešnosť schémy (%)"

plot "data/50000_20_A" smooth csplines dt 1 title 'A', \
     "data/50000_20_B" smooth csplines dt 2 title 'B', \
     "data/50000_20_C" smooth csplines dt 3 title 'C', \
     "data/50000_20_D" smooth csplines dt 4 title 'D', \
     "data/50000_20_E" smooth csplines dt 5 title 'E', \
     "data/50000_20_F" smooth csplines dt 1 title 'F', \
     "data/50000_20_G" smooth csplines dt 2 title 'G', \
     "data/50000_20_H" smooth csplines dt 3 title 'H', \
     "data/50000_20_I" smooth csplines dt 4 title 'I', \
     "data/50000_20_J" smooth csplines dt 5 title 'J'

\end{gnuplot}
\caption{Počet iterácii: 50000, počiatočná populácia: 20}
\end{figure}

%\begin{figure}[!htbp]
\def\svgwidth{\columnwidth}
\centering
\begin{gnuplot}[terminal=pdf,terminaloptions=color]
set terminal pdf enhanced size 15cm, 9cm
set xrange [0:2000]
set yrange [0:1]

set key samplen 3 spacing 1 font ',10' left title 'Schéma'

set xlabel "Počet znakov zašifrovaného textu"
set ylabel "Úspešnosť schémy (%)"

plot "data/ga/50000_50_A" smooth csplines dt 1 title 'A', \
     "data/ga/50000_50_B" smooth csplines dt 2 title 'B', \
     "data/ga/50000_50_C" smooth csplines dt 3 title 'C', \
     "data/ga/50000_50_D" smooth csplines dt 4 title 'D', \
     "data/ga/50000_50_E" smooth csplines dt 5 title 'E', \
     "data/ga/50000_50_F" smooth csplines dt 1 title 'F', \
     "data/ga/50000_50_G" smooth csplines dt 2 title 'G', \
     "data/ga/50000_50_H" smooth csplines dt 3 title 'H', \
     "data/ga/50000_50_I" smooth csplines dt 4 title 'I', \
     "data/ga/50000_50_J" smooth csplines dt 5 title 'J'

\end{gnuplot}
\caption{Závislosť úspešnosti lúštenia od dĺžky ZT (50000 iterácií, 50 jedincov)}
\label{schema:ga_50000_50}
\end{figure}

%\begin{figure}[!ht]
\def\svgwidth{\columnwidth}
\centering
\begin{gnuplot}[terminal=pdf,terminaloptions=color]
set terminal pdf enhanced size 16cm, 10cm
set xrange [0:2000]
set yrange [0:1]

set key samplen 3 spacing 1 font ',10' left title 'Schema'

set xlabel "Počet znakov zašifrovaného textu"
set ylabel "Úspešnosť schémy (%)"

plot "data/50000_100_A" smooth csplines dt 1 title 'A', \
     "data/50000_100_B" smooth csplines dt 2 title 'B', \
     "data/50000_100_C" smooth csplines dt 3 title 'C', \
     "data/50000_100_D" smooth csplines dt 4 title 'D', \
     "data/50000_100_E" smooth csplines dt 5 title 'E', \
     "data/50000_100_F" smooth csplines dt 1 title 'F', \
     "data/50000_100_G" smooth csplines dt 2 title 'G', \
     "data/50000_100_H" smooth csplines dt 3 title 'H', \
     "data/50000_100_I" smooth csplines dt 4 title 'I', \
     "data/50000_100_J" smooth csplines dt 5 title 'J'

\end{gnuplot}
\caption{Počet iterácii: 50000, počiatočná populácia: 100}
\label{schema:ga_50000_100}
\end{figure}


%% Paralelene Geneticke algoritmy
%% Trieda Migrator
%% Schemy PGA
%% Vysledky PGA
%% Spolupraca s Petrom Javorkom
%% Spotrebovany cas/peniaze

